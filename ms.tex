% TODO
% - Sand, Seth, et al. Virgo: blue stars with nothing else
% - Split cluster into 2? Price-Whelan 1a and 1b?

% \documentclass[modern]{aastex62}
\documentclass[twocolumn]{aastex62}
\usepackage{amsmath}
\graphicspath{{./}{figures/}}

% typography
\setlength{\parindent}{1.\baselineskip}
\newcommand{\acronym}[1]{{\small{#1}}}
\newcommand{\project}[1]{\textsl{#1}}
\newcommand{\package}[1]{\textsl{#1}}
\newcommand{\gaia}{\textsl{Gaia}}
\newcommand{\pans}{\acronym{PS1}}
\newcommand{\decam}{DECam}
\newcommand{\DR}[1]{\acronym{DR#1}}
\newcommand{\todo}[1]{{\color{red} TODO: #1}}
\newcommand{\bs}[1]{\boldsymbol{#1}}

\newcommand{\articlename}{\textsl{Letter}}
\newcommand{\sectionname}{Section}
\newcommand{\equationname}{Equation}

% Stats / probability
\newcommand{\given}{\,|\,}
\newcommand{\norm}{\mathcal{N}}

% Maths
\newcommand{\dd}{\mathrm{d}}
\newcommand{\transpose}[1]{{#1}^{\mathsf{T}}}
\newcommand{\inverse}[1]{{#1}^{-1}}
\newcommand{\argmin}{\operatornamewithlimits{argmin}}
\newcommand{\mean}[1]{\left< #1 \right>}
\newcommand{\mat}[1]{\mathbf{#1}}

% Units
\newcommand{\msun}{\textrm{M}_\odot}
\newcommand{\kpc}{\textrm{kpc}}
\newcommand{\kms}{\ensuremath{\textrm{km}~\textrm{s}^{-1}}}
\newcommand{\masyr}{\ensuremath{\textrm{mas}~\textrm{yr}^{-1}}}

% Astronomy
\newcommand{\feh}{\ensuremath{[\textrm{Fe} / \textrm{H}]}}
\newcommand{\vlsr}{$V_{\rm LSR}~$}
\newcommand{\vlsre}{$V_{\rm LSR}$}
\newcommand{\hi}{H{\footnotesize I} }
\newcommand{\hie}{H{\footnotesize I}}
\newcommand{\lms}{$L_{\rm MS}~$}
\newcommand{\lmse}{$L_{\rm MS}$}
\newcommand{\bms}{$B_{\rm MS}~$}
\newcommand{\bmse}{$B_{\rm MS}$}

% Shortcuts for this paper
\newcommand{\clustername}{\textsl{Price-Whelan 1}}
\newcommand{\lmcsmc}{LMC--SMC}
\newcommand{\bprp}{\ensuremath{\textrm{BP} - \textrm{RP}}}
\newcommand{\truepm}{\ensuremath{\tilde{\bs{\mu}}}}
\newcommand{\isodist}{\ensuremath{25~\kpc}}
\newcommand{\eep}{\ensuremath{\acronym{\textrm{EEP}}}}
\newcommand{\Nisofit}{417}

\newcommand{\clage}{\ensuremath{130~\textrm{Myr}}}
\newcommand{\clfeh}{\ensuremath{-1.1}}
\newcommand{\cldist}{\ensuremath{29~\textrm{kpc}}}

%% Reintroduced the \received and \accepted commands from AASTeX v5.2
% \received{January 1, 2018}
% \revised{January 7, 2018}
% \accepted{\today}

% \submitjournal{ApJ}
\shorttitle{A recent star formation event in the Magellanic stream}
\shortauthors{Price-Whelan et al.}

\begin{document}

% \title{A young star cluster in the Galactic halo: \\
% Recent star formation associated with the Leading Arm of the Magellanic stream}
\title{\textbf{Discovery of a disrupting open cluster in the Milky Way halo:\\
       a recent star formation event in the leading arm of the Magellanic stream?}}

\author[0000-0003-0872-7098]{Adrian~M.~Price-Whelan}
\affiliation{Department of Astrophysical Sciences,
             Princeton University, Princeton, NJ 08544, USA}
\email{adrn@astro.princeton.edu}
\correspondingauthor{Adrian M. Price-Whelan}

\author[0000-0002-1793-3689]{David~L.~Nidever}
\affiliation{Department of Physics, Montana State University, P.O. Box 173840, Bozeman, MT 59717, USA}
\affiliation{National Optical Astronomy Observatory, 950 North Cherry Ave, Tucson, AZ 85719, USA}

\author[0000-0003-1680-1884]{Yumi~Choi}
\affiliation{Department of Physics, Montana State University, P.O. Box 173840, Bozeman, MT 59717, USA}
\affiliation{Steward Observatory, University of Arizona, 933 North Cherry Avenue, Tucson AZ, 85721, USA}

\author[0000-0002-3569-7421]{Edward~F.~Schlafly}
\affiliation{Lawrence Berkeley National Laboratory, One Cyclotron Road, Berkeley, CA 94720, USA}

\author[0000-0003-2808-275X]{Douglas~P.~Finkbeiner}
\affiliation{Institute for Theory and Computation, Harvard-Smithsonian Center for Astrophysics, 60 Garden Street, MS-51, Cambridge, MA 02138, USA}

\author[0000-0002-8537-5711]{Timothy~Morton} % TODO: check orcID
\affiliation{Department of Astrophysical Sciences,
             Princeton University, Princeton, NJ 08544, USA}

\author[0000-0003-2644-135X]{Sergey E. Koposov}
\affiliation{McWilliams Center for Cosmology, Department of Physics, Carnegie Mellon University, 5000 Forbes Avenue, Pittsburgh, PA 15213, USA}
\affiliation{Institute of Astronomy, University of Cambridge, Madingley Road, Cambridge CB3 0HA, UK}

\author[0000-0002-0038-9584]{Vasily Belokurov}
\affiliation{Institute of Astronomy, University of Cambridge, Madingley Road, Cambridge CB3 0HA, UK}
\affiliation{Center for Computational Astrophysics, Flatiron Institute, 162 5th Avenue, New York, NY 10010, USA}


\begin{abstract}

We report the discovery of a young $\tau \sim \clage$, low-mass $M \sim 1200~\msun$, metal-poor $[\textrm{Fe}/\textrm{H}] \sim \clfeh$ stellar association in the Milky Way halo, \clustername.
At a Galactocentric (cylindrical) position $(R, z) \sim (23, 15)~\textrm{kpc}$, the association is (on the sky) near the leading arm of the gas stream emanating from the Magellanic cloud system, but is located $\approx 60^\circ$ from the Large Magellanic Cloud (LMC) center, on the other side of the Milky Way disk relative to the LMC.
By assuming that the cluster is co-located with \hi gas in the stream, we directly measure the distance to the Magellanic stream, $D \approx \cldist$.
At this location relative to the LMC, the measured distance is inconsistent with predictions from models of the LMC/SMC interaction and infall into the Milky Way that do not account for ram pressure and gas interaction with MW disk.
The estimated age of \clustername\ is consistent with the time of last passage through the Galactic midplane, and we therefore speculate that this star-formation event was triggered by the last disk midplane passage of the Magellanic stream gas, which occurred at a Galactocentric radius $R \approx 30~\kpc$.
However, most details of this idea remain a mystery: the Magellanic stream is low column density, the Milky Way disk at this large radius has low surface density, and the relative velocity of the stream gas and MW gas is large.
However it formed, the discovery of a young stellar cluster in the Milky Way halo presents an interesting opportunity for study.
\end{abstract}

\keywords{
    Galaxy: open clusters and associations --
    Galaxy: halo --
    stars: formation --
    surveys}


\section{Introduction} \label{sec:intro}

% The halo of the Milky Way is characterized by its old ($\gtrsim 10~\textrm{Gyr}$), metal-poor ($\feh \approx -1.5$) stellar population.
% This is understood as a signature of the dominant (in stellar mass) progenitor systems that were accreted and disrupted early on in the formation of the Galaxy: massive dwarf galaxies \citep[e.g.,][]{Deason:2015, Fiorentino:2015}.
% These systems likely came in with significant gas reservoirs, but were quenched and stripped through collisional processes that heated and dispersed the gas \citep[e.g.,][]{Mayer:2006}, thus preventing immediate star formation in the deposited gas.
% The Milky Way, however, continues to accrete satellite galaxies, as is evidenced by the prominent stellar stream from the Sagittarius dwarf galaxy \citep{Ibata:1994, Majewski:2003}, the presence of the Large and Small Magellanic Clouds (LMC, SMC), and about 50 dwarf satellites within the virial radius of the Galaxy.
% While Sagittarius was likely stripped of its neutral gas long ago through repeated passage about the Milky Way \citep{Burton:1999, Tepper-Garcia:2018}, the \lmcsmc\ system is associated with $\approx 8\times 10^8~\msun$ of \hi gas \citep{Bruns:2005}, which extends into leading and trailing gas streams that wrap nearly $\approx 200^\circ$ around the sky \citep{Mathewson:1974, Putman:1998, Bruns:2005, Nidever:2010}.

The Magellanic stream (MS) is a large stream of (predominantly) hydrogen gas emanating from the \lmcsmc\ system that contains a significant fraction of the total gas mass associated with the \lmcsmc\ \citep{Bruns:2005} and wraps nearly $\approx 200^\circ$ around the sky \citep{Mathewson:1974, Putman:1998, Bruns:2005, Nidever:2010}.
The stream was discovered in early surveys of 21 cm emission and has since been studied in great detail by large-area radio sky surveys, and in H$\alpha$ emission \citep{Weiner:1996, Barger:2017}.
With increased resolution, surveys have found small-scale structure and gas fragmentation \citep[e.g.,][]{Nidever:2008, For:2014} and a large-scale bifurcation, with %semi-distinct ``strands'' that lead back to the LMC and broader system, respectively \citep[e.g.,][]{Morras:1983, Putman:2003}.
kinematically \citep{Nidever:2008} and chemically \citep{Fox:2013} distinct ``strands'' that lead back to the LMC and the SMC, respectively.
As surveys push deeper to lower column densities, extensions of the stream have been found both in the trailing region \citep{Nidever:2010}, but also in the leading arm, which has been found to connect to regions of low-column-density gas ($N\sim 10^{18}$--$10^{19}~\textrm{cm}^{-2}$) on the other side of the Galactic disk \citep{Putman:1998, Nidever:2008}.
The leading arm (LA) has been decomposed into three or four distinct ``features'' named LA1--4 \citep{Bruns:2005, Nidever:2008, Venzmer:2012} with total HI mass $\sim 4 \times10^7~\msun$ \citep{Venzmer:2012}.

The origin and formation of the LA features in the MS is still uncertain.
Initial studies of the LA argued that the features closest to the \lmcsmc\ can be traced back to the SMC, thus implying that the gas in the LA was stripped from the SMC \citep{Putman:1998}.
In later work, it has instead been argued that outer features of the LMC appear to lead directly into the LA-1 feature \citep{Nidever:2008}, implying an LMC origin for the leading arm.
Recent chemical abundance measurements along several sightlines passing through the LA again suggested the SMC origin \citep{Fox:2018, Richter:2018}.
Recent hydrodynamical simulations of the MCs on the first infall to the Milky Way, however, showed that the repeated encounters between the MCs strip gas not only from the SMC, but also from the LMC, and this tidally stripped gas creates filamentary gas structures both in the LA and the Magellanic stream \citep{Pardy:2018}.
Whatever the origin of the gas, it is clear that tidal stripping by the Milky Way is required to form the leading arm \citep{Nidever:2008, Besla:2012}.
However, the LA features deviate from the predicted orbit of the \lmcsmc, implying that ram pressure or interactions with gas in the outer Milky Way disk may have removed orbital energy from the LA gas \citep[e.g.,][]{Bekki:2008}.

Streams formed from tidally stripped material --- such as the Sagittarius or Palomar 5 \citep{Odenkirchen:2001} stellar streams, or gas streams such as the MS --- encode information about the orbital history and future trajectory of their progenitor system \citep[e.g.,][]{Johnston:1999}.
For this reason, streams of debris are of interest for constraining the dark matter distribution around the Galaxy, as well as for providing a fossil record of the Milky Way's accretion history \citep{Johnston:1998, Bullock:2005}.
Several groups have attempted to use the MS to constrain both the infall history of the \lmcsmc\ and the dark matter halo of the Milky Way (see recent review by \citealt{DOnghia:2016}).
Early models for the formation of the MS invoke ram pressure and tidal stripping to explain the sky distribution and extent of the MS, and typically require that the \lmcsmc\ system complete several orbits around the Galaxy in order to strip enough material \citep[e.g.,][]{Murai:1980, Gardiner:1994, Mastropietro:2005}.
However, later measurements of the proper motion of the LMC (\citealt{Kallivayalil:2006}, later confirmed by \gaia\ \citealt{van-der-Marel:2016}) and improved models for the infall dynamics \citep{Besla:2007} suggest instead that the \lmcsmc\ system could be on its first passage through the Galaxy \citep{Besla:2010, Besla:2012}.
In addition, the orbit of the LMC implied by the new proper motion measurement appears to be offset by $\approx 10^\circ$ from the trailing MS \citep{Besla:2010}.
Put together, these recent revelations have led to new models for the formation of the MS that largely rely on past interactions between the LMC and SMC to preprocess the gas distribution before infall and eventual stripping by the tidal field of the Milky Way \citep{Besla:2012, Diaz:2012}.

One critical difficulty in using the MS to constrain models of the \lmcsmc\ interaction and infall is the lack of distance information along the stream.
No significant over-density of stars have been found associated with the trailing portion of the MS \citep{Guhathakurta:1998}, thus leaving distance and tangential velocity information largely unknown.
Recently, a small number of young OB stars have been found in the vicinity of the leading arm gas, with radial velocities consistent with having formed from the leading MS \citep{Casetti-Dinescu:2014, Zhang:2017}.
However, given their sparsity and concentration near the Galactic plane, it is difficult to unambiguously associate them with the MS: these could instead be runaway OB stars from the Milky Way disk.

In this \articlename, we report the discovery of a clustered association of young stars at the far edge of the LA-2 feature ($L_{\textrm{MS}} \sim 65^\circ$) that is unambiguously located far into the Galactic halo ($D \sim \cldist$) and therefore plausibly formed from gas in the leading arm of the Magellanic stream as it crossed the Galactic disk.
This provides a clear distance measurement to the leading arm of the MS to be compared with models of \lmcsmc\ interaction and infall into the Milky Way, and also provides an opportunity to study recent star formation in a regime unlike that of the Milky Way.
The discovery of \clustername\ will enable new modeling efforts that track the infall of the LMC-SMC, the tidal stripping of Magellanic gas, and the interaction of this gas with the Milky Way.

In \sectionname~\ref{sec:data}, we present the initial discovery with \gaia\ \DR{2} and follow-up observations with \decam\ to obtain deeper photometry of the region around the association.
In \sectionname~\ref{sec:results}, we use the \decam\ photometry to infer the age, metallicity, and distance to \clustername, and discuss plausible formation scenarios for this cluster.
% In \sectionname~\ref{sec:discussion}, we discuss implications for future searches for stars associated with the MS, and discrepancies with existing models of the LA.
We conclude in \sectionname~\ref{sec:conclusion}.

% Notebook: Figure-CMDs
\begin{figure*}[htb!]
\centering
\includegraphics[width=\textwidth]{figures/gaia-cmd-pm.pdf}
\caption{Color-magnitude diagram, sky positions, and proper motions from \gaia\ \DR{2} for the region around $(\alpha, \delta) \sim (179, -29)^\circ$.
\textbf{Left panel}: \gaia\ color-magnitude diagram (CMD) for all sources with $G < 20$ and sky position within 3 degrees of $(\alpha, \delta) = (179.5, -28.8)^\circ$.
Red line shows a $100~\textrm{Myr}$, $\feh = -1.1$ \acronym{MIST} isochrone shifted to a distance of $30~\kpc$.
Shaded pixels (orange) show regions of high density and points (black markers) are plotted when there are fewer than 8 stars per pixel.
\textbf{Middle panel}: Sky positions for all sources in the shown sky region (grey 2D histogram).
Black markers show only sources in the blue selection polygon shown in the CMD (left panel).
\textbf{Right panel}: Proper motions for all sources in the shown sky region (grey 2D histogram).
Black markers again show only sources in the blue CMD selection region (left panel).
}
\label{fig:cmds}
\end{figure*}

\section{Data} \label{sec:data}

\subsection{Cluster discovery with \gaia}
\label{sec:discovery}

We use astrometric data from the \gaia\ mission (\citealt{Prusti:2016}), data release 2 (\DR{2}; \citealt{Gaia-Collaboration:2018, Lindegren:2018}) to search for distant, comoving multiplets of blue stars.
Our original intent was to search for small, distant, comoving groups of blue horizontal branch stars to identify new candidate satellites of the Milky Way.
We therefore initially select all stars from \gaia\ with parallax $\varpi < 1$, color $-0.5 < (\bprp) < 0$, $G$-band magnitude $G < 20$, and Galactic latitude $|b| > 20^\circ$ (see Appendix~\ref{sec:queries} for database query).
We further exclude stars within a $15^\circ$ radius from the LMC, and a $8^\circ$ radius from the SMC --- 27,895 stars remain after these cuts.
We then cross-match this catalog to itself with both sky positions and proper motions: we search for pairs of stars that have separations $s < 0.5^\circ$ and proper motion differences $|\Delta \mu| < 0.5~\masyr$.
We then combine mutually-connected comoving pairs into small groups of stars that are colocated on the sky and comoving in proper motions, and remove groups that have $<4$ members.
We cross-match the mean sky positions of the groups to locations of local group galaxies \citep{McConnachie:2012} and Milky Way globular clusters \citep[2010 edition;][]{Harris:1996} and filter out all groups that lie within 1 degree of these known objects.
After these filters, just one group of blue, comoving stars remains at $(\textrm{RA}, \textrm{Dec}) \sim (179, -29)^\circ$.

We then query all objects from the \gaia\ \DR{2} catalog within a rectangle centered on the nominal position of this group, with a width of $5^\circ$ and a height of $5^\circ$ in the equatorial (ICRS) coordinate system (see Appendix~\ref{sec:queries} for the database query).
\figurename~\ref{fig:cmds} shows the \gaia\ data for this region.
The left panel of \figurename~\ref{fig:cmds} shows the \gaia\ color-magnitude diagram, with the blue over-density highlighted by the polygon (blue) and under-plotted with a $100~\textrm{Myr}$, $\feh = -1.1$ \acronym{MIST} isochrone \citep[red line;][]{Dotter:2016, Choi:2016, Paxton:2011, Paxton:2013, Paxton:2015}: Surprisingly, this appears to be a young, distant main sequence, rather than an old population of horizontal branch stars.
The middle and right panels of \figurename~\ref{fig:cmds} show sky positions and proper motions, respectively, of all stars in this sky region (grey background density), and only stars in the CMD selection polygon as black markers.

The \gaia\ data demonstrate the presence of a young, distant, spatially-clustered, and co-moving stellar over-density --- named \clustername\ --- but the \gaia\ photometry is too shallow to resolve anything but the brightest main sequence stars.
With the \gaia\ data alone, the distance, age, and metallicity of the cluster therefore cannot be determined, as these quantities are degenerate where the main sequence is nearly vertical.
In the next section, we describe deeper \decam\ imaging obtained over a portion of the cluster.


\subsection{Follow-up with \decam}
\label{sec:decam}

We obtained \decam\ $u$-, $g$-, and $i$-band imaging of a single field centered on a portion of the cluster discovered using the \gaia\ data (see previous section).
\todo{Nidever: can you describe the observations? Exposure times, dither, whatever is usually done...}

The photometry was reduced...\todo{Nidever: describe your pipeline?}

\figurename~\ref{fig:decam-field} shows the sky positions of the \gaia\ CMD-selected sources (black markers; see same in middle panel, \figurename~\ref{fig:cmds}), along with point sources identified in the $g$-band data obtained with \decam: sources in control fields are shown as red markers, and sources in cluster fields are shown as blue markers.
We note that the smaller over-density located northeast of the \decam\ field was not followed up in this work.

\figurename~\ref{fig:decam-cmd} shows \decam\ color-magnitude diagrams for the control and cluster sub-fields (the magnitudes here are not extinction-corrected).
We find that the $u$-band data were too shallow to be useful for this work, so restrict the further analysis in this work to the $g$- and $i$-bands.
The blue comoving group, \clustername, identified in \sectionname~\ref{sec:discovery} using \gaia\ data alone shows up as a clear young main sequence in the cluster fields (right panel of \figurename~\ref{fig:decam-cmd}).
Later, we use photometry in the sub-region of the cluster fields identified by the dashed rectangle to infer the cluster parameters.

 % Notebook: Figure-DECam-CMD
 \begin{figure}
 \centering
 \includegraphics[width=0.45\textwidth]{figures/DECam-field.pdf}
 \caption{The sky region around \clustername, showing the same \gaia\ CMD-selected sources as in \figurename~\ref{fig:cmds} (black markers), and point sources identified from \decam\ $g$-band follow up (blue and red markers).
 Red markers show sources in the \decam\ field used as control sources, and blue markers show sources used as cluster member candidates.}
 \label{fig:decam-field}
 \end{figure}

% Notebook: Figure-DECam-CMD
\begin{figure}
\centering
\includegraphics[width=0.45\textwidth]{figures/DECam-cmd.pdf}
\caption{\decam\ $g-i$ versus $g$ color-magnitude diagrams for the \decam\ control fields (left panel) and \decam\ cluster fields (right panel).
Note the prominent young, distant main sequence in the cluster fields: This is the main sequence of \clustername.
The sources in the dashed (red) rectangle in the right panel are later used to measure the cluster stellar population parameters.
}
\label{fig:decam-cmd}
\end{figure}


\section{Methods} \label{sec:methods}

In the subsections below, we perform two independent analyses of the data available for \clustername.
First, in \sectionname~\ref{sec:pmclean}, we use astrometric data from \gaia\ \DR{2} to determine the mean proper motion of the cluster by modeling the kinematics of the cluster and background sources.
Then, in \sectionname~\ref{sec:popmodel}, we use photometric data from \decam\ to assign membership probabilities to sources and simultaneously infer the cluster stellar population parameters (age, metallicity, distance, etc.).
We perform these analyses separately because \gaia\ \DR{2} only includes the most massive members of the cluster, but the addition of lower main sequence members apparent in the \decam\ photometry provide a much better constraint on the cluster parameters.


\subsection{Inferring the mean proper motion}
\label{sec:pmclean}

% Notebook: Proper-motion-membership
\begin{figure*}[ht!]
\centering
\includegraphics[width=\textwidth]{figures/pm-model.pdf}
\caption{\textbf{Left panel}: Sky positions of all \gaia\ sources in this region with $\bprp < 0.35$ (grey markers).
The circles show the regions used as cluster and control fields in the proper motion inference (\sectionname~\ref{sec:pmclean}).
The control fields, together, have the same area as the cluster field.
\textbf{Right panel}: The colored 2D density shows the difference in number of sources per pixel in the cluster field versus the two control fields (see left panel).
The clear over-density in the cluster field (blue) are stars in \clustername.}
\label{fig:pm-members}
\end{figure*}

We infer the mean proper motion of \clustername\ by constructing a probabilistic model of the cluster and background populations using astrometric data from \gaia.
We start by selecting all stars with $(\bprp)_0 < 0.35$ to remove low-mass and old main sequence star contamination in the region.
\figurename~\ref{fig:pm-members}, left (grey points), shows the sky positions of stars that pass this blue cut in the region around the young cluster.
The larger solid-line circle indicates the region we define as the cluster area, and the two smaller dashed-line circles indicate control fields that are combined and used for modeling the background distribution of proper motions.
The control fields are designed to, together, have the same total area as the cluster field.
The right panel of \figurename~\ref{fig:pm-members} shows the difference of the 2D proper motion distributions in the cluster and summed control fields.
The distinct over-density of stars near $(\mu_\alpha, \mu_\delta) \approx (-0.5, 0.5)~\masyr$ is the identified comoving association of blue stars.\footnote{Throughout this article, we use $\mu_\alpha$ to refer to the proper motion value provided by \gaia, which includes the $\cos\delta$ term.}
Note that \figurename~\ref{fig:pm-members} is only meant an illustration: The cluster proper motion is determined probabilistically by taking into account the full \gaia\ covariance matrices for each source, as described below.

To measure the cluster (mean) proper motion, we first construct a model for the error-deconvolved proper motion distribution in the control fields using ``extreme deconvolution'' \citep[XD;][]{Bovy:2011} with two Gaussian components. XD takes into account the full error distributions for each proper motion measurement $\bs{\mu} = (\mu_\alpha, \mu_\delta)$, including covariances $\mat{C}_\mu$, provided by \gaia\ \DR{2}.
After running XD, we fix the parameters of the background model and model the proper motion distribution in the cluster region using a two-component mixture model with a single, isotropic Gaussian component for the error-deconvolved cluster distribution with mean $\bs{x}$ and isotropic variance $s^2$, and the XD-inferred model for the control field, $p_{\textrm{XD}}$, for the background component.
In detail, taking $f$ to be the fraction of blue stars in this region belonging to the young cluster, and \truepm\ to be the true proper motion for a single star,
\begin{align}
    p(\bs{\mu} \given \truepm, \mat{C}_\mu) &=
        \norm(\bs{\mu} \given \truepm, \mat{C}_\mu)\\
    p(\truepm \given f, \bs{x}, s) &=
        f \, p_{\textrm{cl}}(\truepm \given \bs{x}, s)
        + (1-f) \, p_{\textrm{XD}}(\truepm)\\
    p_{\textrm{cl}}(\truepm \given \bs{x}, s) &=
        \norm(\truepm \given \bs{x}, s^2 \, \mathbb{I})
\end{align}
where $\norm(\cdot \given \bs{y}, \mat{C})$ represents the multidimensional normal distribution with mean $\bs{y}$ and covariance matrix $\mat{C}$, and $\mathbb{I}$ is the identity matrix.
Because all distributions are Gaussian, the per-star parameters (the true proper motions, \truepm) can be analytically marginalized out so that the per-star likelihood can be expressed as $p(\bs{\mu} \given f, \bs{x}, s, \mat{C}_\mu)$.
We assume that the measurements for each star, $n$, are independent so that the full likelihood of all $N$ stars given the parameters $(f, \bs{x}, s)$ is
\begin{align}
    p(\{\bs{\mu}_n\}_N \given f, \bs{x}, s, \{\mat{C}_{\mu, n}\}_N) &=
        \prod_n^N p(\bs{\mu}_n \given f, \bs{x}, s, \mat{C}_{\mu, n}) \quad .
        \label{eq:likelihood}
\end{align}

We use an ensemble Markov chain Monte Carlo (MCMC) sampler \citep[\texttt{emcee};][]{emcee, Goodman:2010} to generate posterior samples over the parameters $(f, \bs{x}, s)$ using the likelihood defined above (\equationname~\ref{eq:likelihood}), and assuming the following prior probability distributions: uniform over the domain $(-5, 5)~\masyr$ for each component of $\bs{x}$, uniform in $f$, and uniform in log-$s$ over the domain $-6 < \ln s < 4$ (with $s$ in units of \masyr).
We run the sampler with 32 walkers for 256 steps as burn-in, then reset the sampler and run for an additional 512 steps.
We then thin the resulting chains by taking every 16th sample, leaving a total of 1024 samples; We use these samples to estimate the median posterior parameter values and uncertainties,
For the young cluster, we find $\bs{x} = (-0.56,  0.47) \pm (0.04, 0.02)~\masyr$, $\ln s = -3.8 \pm 0.9$, and $f = 0.14 \pm 0.02$.
The proper-motion dispersion is consistent with zero, and is likely unmeasured; 95\% of the posterior samples have $s < 0.09~\masyr$.


\subsection{Inferring the stellar population parameters}
\label{sec:popmodel}

While the \gaia\ data provide exquisite astrometric data for the brightest members of \clustername, constraints on the stellar population from the \gaia\ data alone are limited by degeneracies in the mapping from the color-magnitude diagram to stellar parameters.
To measure the cluster population parameters, we therefore use the deeper photometry from \decam, which provides a clear view of the main sequence of \clustername\ down to stellar masses $M \sim 0.9~\msun$.
\figurename~\ref{fig:decam-cmd} shows the \decam\ $g$ and $i$ band color-magnitude diagrams for control fields (left) and cluster fields (right) selected from the \decam\ footprint, once again showing the young main sequence (coherent stellar population in right panel).
We use the $g$- and $i$-band photometry for individual stars in a sub-section of the cluster field CMD (red dashed outlined region in \figurename~\ref{fig:decam-cmd}) to infer the stellar population parameters of \clustername.
To summarize our methodology, we first generate independent posterior samplings over the stellar parameters of each individual source under an interim prior (over age, stellar mass, distance, etc.), then use these individual samplings to construct a Bayesian hierarchical model for a two-component mixture model of the cluster and a background population.
If you prefer to skip the details, the results of this modeling are presented in \sectionname~\ref{sec:popchars}.

In detail, we start by using the \texttt{isochrones} package \citep{Morton:2015} to generate posterior samplings over stellar parameters for each individual source given its photometry and an interim prior.
We use the \acronym{MIST} \citep{Dotter:2016, Choi:2016, Paxton:2011, Paxton:2013, Paxton:2015} isochrone grid, and \texttt{isochrones} automatically performs interpolation between the provided grid of stellar isochrones to predict photometry given a set of stellar parameters.
Here, the stellar parameters for each source are its ``equal evolutionary point'' number $\eep$ (see \acronym{MIST} documentation\footnote{\url{http://waps.cfa.harvard.edu/MIST/}}), age $\tau$, metallicity $\feh$, extinction $A_V$, and distance $D$, which can be uniquely mapped to a point in the observed CMD;
The likelihood of a given set of these parameters, $\bs{\theta} = (\eep, \tau, \feh, A_V, D)$, is then computed from the photometry and (assumed Gaussian) photometric uncertainties given the predicted photometry.

To generate posterior samples, we must also specify prior probability distributions over the parameters $\bs{\theta}$.
The priors for each parameter are summarized in \tablename~\ref{tbl:priors}.
The bounds on $\eep$ limit the isochrone to evolutionary phases between the zero-age main sequence to the terminal-age main sequence.
The prior and bounds on $A_V$ are set to prefer small and reasonable values of extinction for this moderately high Galactic latitude region.
The prior on distance assumes a uniform space density of stars.

\begin{table}[ht!]
\begin{center}
    \begin{tabular}{ c | c | c }
        \toprule
        Parameter & Prior & Bounds \\
        \toprule
        $\eep$ & uniform & $(202, 355)$ \\
        $\tau$ & uniform & $(10~\textrm{Myr}, 15~\textrm{Gyr})$ \\
        $\feh$ & uniform & $(-2, 0.5)$ \\
        $A_V$ & $\propto A_V^{-1}$ & $(0.001, 1)~\textrm{mag}$ \\
        $D$ & $\propto D^{2}$ & $(1, 100)~\kpc$ \\
        \toprule
    \end{tabular}
\caption{Prior probability distributions for each of the stellar parameters defined in \sectionname~\ref{sec:popmodel}.
\label{tbl:priors}}
\end{center}
\end{table}

We find that the photometry for individual sources in the lower main sequence are very poorly constrained in all parameters, and the prior tends to pull the posterior samplings to prefer closer, older stellar parameters.
This is a weakness of our methodology for performing the hierarchical inference: Each source is considered independently, even though there is clear structure in the CMD (i.e. \figurename~\ref{fig:decam-cmd}), and inferring isochronal parameters for individual lower main sequence stars is a fundamentally degenerate problem.
A more correct way to do this would be to infer the stellar parameters of \emph{all} stars, the cluster hyperparameters, and the background simultaneously.
However, for the \Nisofit\ stars we are using (in the red box in \figurename~\ref{fig:decam-cmd}), this model would have $\sim$2,000 free parameters if left unmarginalized.
We are developing tools to perform star cluster parameter inference in this way (Morton et al., in prep.), but here we adopt a simple hack to allow us to instead perform individual posterior samplings and then combine those samplings into a hierarchical inference.

The brightest stars in the cluster have very precise \decam\ photometry, and have precise astrometry from \gaia.
Given their location in the CMD, these stars must be young, and so also have much better constrained stellar parameters.
We therefore use the brightest \emph{kinematic} member of \clustername\ as an ``anchor'' star:
We model every other source in the \decam\ selection region (i.e. excluding this one) by fitting the photometry of it and the anchor star simultaneously, assuming they have the same age, metallicity, distance, and extinction but different \eep\ values.
Motivated by the possible signature of an (unresolved) binary sequence in the \decam\ CMD, we add one further piece of complexity to the model by allowing the photometry of each non-anchor source to be fit as an unresolved binary star system.
This adds an additional parameter, the unresolved binary mass ratio $q$, to the list of inferred stellar parameters for each source, but, for this work, we ignore the binary companions and implicitly marginalize over $q$ in what follows.
The details of this model are handled by the \texttt{isochrones} package.

We generate posterior samplings over the parameters $(\bs{\theta}, q)$ for each of the \Nisofit\ sources in the selected region of the \decam\ CMD using \texttt{PyMultinest} \citep{Buchner:2014, Feroz:2008, Feroz:2009}, and store the value of the prior evaluated at the location of each sample.
We then use these samples and interim prior values to construct our hierarchical model, as described below.

In the hierarchical model, we assume that the stellar parameters of each primary star in the selected region are either drawn from the cluster, or a background (stellar halo) population.
For the cluster, we assume that the values are drawn from delta functions in age, metallicity, distance, and extinction, with the centroids of the delta functions $\bs{\alpha} = (\tau^*, \feh^*, D^*, A_V^*)$ as hyperparameters of the hierarchical inference.
For the background model, we assume the same priors as specified in \tablename~\ref{tbl:priors}.
The one additional parameter that must be included in this hierarchical model is the mixture weight: The global fraction of sources that are likely cluster members, $f$.
To compute the likelihood for the hierarchical model, we use the individual posterior samplings to marginalize over the per-source stellar parameters $\bs{\theta}_n$ to compute the marginal likelihood $p(\bs{m}_n \given \bs{\alpha}, f)$, where $\bs{m}_n = (g, i)_n$ is the vector of photometric data for source $n$;
This likelihood for a single source given a set of hyperparameters $(\bs{\alpha}, f)$ is then
\begin{equation}
    p(\bs{m}_n \given \bs{\alpha}, f) = \int \dd \bs{\theta}_n \,
        p(\bs{m}_n \given \bs{\theta}_n) \,
        p(\bs{\theta}_n \given \bs{\alpha}, f) \quad .
\end{equation}
We employ the ``importance sampling trick'' (see, e.g., Appendix of \citealt{Price-Whelan:2018} or \citealt{Hogg:2010, Foreman-Mackey:2014} for other examples) to re-write an approximate form for this marginal likelihood as
\begin{equation}
    p(\bs{m}_n \given \bs{\alpha}, f) \approx \frac{\mathcal{Z}_n}{K} \,
        \sum_k^K \frac{p(\bs{\theta}_{nk} \given \bs{\alpha}, f)}{p(\bs{\theta}_{nk} \given \bs{\alpha}_0, f)}
        \label{eq:marglike}
\end{equation}
where the index $k$ specifies the index of one of $K$ posterior samples generated from the independent samplings (described above), $\mathcal{Z}_n$ is a constant, and the denominator, $p(\bs{\theta}_{nk} \given \bs{\alpha}_0, f)$, are the values of the interim prior used to do the independent samplings.
In this work, $N=\Nisofit$ and we adopt $K=2048$.

With the marginal likelihood (\equationname~\ref{eq:marglike}), we then need to specify prior probability distributions for the hyperparameters $(\bs{\alpha}, f)$, and we can then generate posterior samples for the hyperparameters.
We use uniform priors for all of these, as summarized in \tablename~\ref{tbl:hyperpriors}.
We use \texttt{emcee} \citep{emcee, Goodman:2010} to sample from the posterior probability distribution for the hyperparameters given all of the photometric data,
\begin{equation}
    p(\bs{\alpha}, f \given \{\bs{m}_n\}) \propto
        p(\bs{\alpha}) \, p(f) \,
        \prod_n^N p(\bs{m}_n \given \bs{\alpha}, f) \quad .
\end{equation}
Here we use 64 walkers and run for an initial 128 steps to burn-in the sampler before running for a final 1024 steps.
\figurename~\ref{fig:hierarch-corner} shows a corner plot with all 1D and 2D marginal posterior probability distributions estimated from the samples.

\begin{table}[ht]
\begin{center}
    \begin{tabular}{ c | c | c }
        \toprule
        Parameter & Prior & Bounds \\
        \toprule
        $\tau^*$ & uniform & $(1~\textrm{Myr}, 1~\textrm{Gyr})$ \\
        $\feh^*$ & uniform & $(-2, 0)$ \\
        $A_V^*$ & uniform & $(0, 1)~\textrm{mag}$ \\
        $D^*$ & uniform & $(1, 100)~\kpc$ \\
        $f$ & uniform & (0, 1) \\
        \toprule
    \end{tabular}
\caption{Prior probability distributions for the hyperparameters $(\bs{\alpha}, f)$.
\label{tbl:hyperpriors}}
\end{center}
\end{table}

% TODO: Make a PGM for this model?

% Notebook: Figure-hierarch-results
\begin{figure}
\centering
\includegraphics[width=0.48\textwidth]{figures/hierarch-corner.pdf}
\caption{A corner plot showing marginal posterior probability distributions estimated from the the posterior samples generated from the hierarchical inference of the cluster population parameters.}
\label{fig:hierarch-corner}
\end{figure}


\section{Results} \label{sec:results}

\subsection{The stellar population and physical characteristics of \clustername}
\label{sec:popchars}

We use the posterior samples from the hierarchical inference described in \sectionname~\ref{sec:popmodel} to compute the posterior probabilities that each source in the \decam\ CMD selection box (red box, \figurename~\ref{fig:decam-cmd}) is a member of the cluster.
\figurename~\ref{fig:hierarch-iso} shows the \decam\ photometry for all sources with membership probability $> 0.5$ (circle markers).
Over-plotted in \figurename~\ref{fig:hierarch-iso} (blue, solid line) is a \acronym{MIST} isochrone with the median posterior parameters derived from the hierarchical modeling, shifted to the median distance and extincted given the median $A_V$ value.
We find that \clustername\ is indeed young, distant, and metal poor, with median posterior values of age $\tau \approx \clage$, distance $D \approx \cldist$, and metallicity $\feh \approx \clfeh$ (see also \tablename~\ref{tbl:clusterparams} for a summary).
Also plotted in \figurename~\ref{fig:hierarch-iso} is the equal-mass binary sequence (green, dashed line) computed from the median posterior sample: The abundance of sources between the nominal isochrone and the binary sequence highlights the fact that the cluster may contain a significant number of binary or multiple star systems, but we leave a detailed study of multiplicity to future work.

We use the isochrone corresponding to the median posterior sample to estimate the total stellar mass of the cluster.
By assuming that the \decam\ imaging is 100\% complete to stars with $(g-i) < 0.3)$ and $g < 22$, and by assuming a Kroupa initial mass function \citep{Kroupa:2001}, we use the number of observed stars and the isochrone to compute the total mass, $M_{\rm tot, *} \approx 1200~\msun$.

The mass and age of \clustername\ are comparable to Milky Way disk open clusters, but with a much lower metallicity, and a much larger spatial extent.
For example, the Pleiades has an age $\sim 135~\textrm{Myr}$ \citep{Gossage:2018}, but a physical size $\sim 5~\textrm{pc}$.
At a distance of $\cldist$, \clustername\ spans $\sim 1.5^\circ$, corresponding to a physical size $\sim 700$--$800~\textrm{pc}$.
If it formed unbound, but with an initial size comparable to the present size of the Pleiades, this corresponds to an expansion velocity $\sim 6~\kms$, which would be detectable with precise radial velocity measurements of stars on either side of \clustername.

% Notebook: Figure-hierarch-results
\begin{figure}
\centering
\includegraphics[width=0.48\textwidth]{figures/hierarch-results.pdf}
\caption{The \decam\ color-magnitude diagram for sources with probabilty $>0.5$ of belonging to \clustername\ (markers).
The solid line (blue) shows the \acronym{MIST} isochrone for the median posterior sample from the hierarchical inference of the cluster population parameters, shifted to the inferred distance to \clustername, and extincted with the inferred extinction.
The dashed line (green) shows the same isochrone, shifted $\approx 0.75$ magnitudes brighter, representing the expected location of the equal-mass binary sequence of the cluster.
}
\label{fig:hierarch-iso}
\end{figure}

\begin{table}[ht]
\begin{center}
    \begin{tabular}{ r | l | l}
        \toprule
        Parameter & Value & Description \\
        \tableline
        $\mean{\alpha}$ & $178.8^\circ$ & mean right ascension \\
        $\mean{\delta}$ & $-29.4^\circ$ & mean declination \\
        $\mean{D}$ & $29~\kpc$ & mean Heliocentric distance \\
        $\mean{\mu_\alpha}$ & $-0.56~\masyr$ & mean proper motion in RA\\
        $\mean{\mu_\delta}$ & $0.47~\masyr$ & mean proper motion in Dec\\
        \tableline
        $\tau$ & $130~\textrm{Myr}$ & age \\
        $M$ & $1200~\msun$ & total mass \\
        $\feh$ & $-1.1$ & metallicity \\
        \tableline
        $A_V$ & $0.3$ & extinction \\
        \toprule
    \end{tabular}
\caption{Inferred parameters for \clustername.
\label{tbl:clusterparams}}
\end{center}
\end{table}


\subsection{Relation to the Magellanic stream}
\label{sec:higas}

At the sky location and distance of \clustername, i.e. well into the Galactic halo, the only plausible gas reservoirs that could have formed a young cluster are the Magellanic stream, or a previously unknown high velocity cloud (HVC).
HVCs are thought to either be accreted and therefore lower metallicity than typical present-day Milky Way gas, or ejected from the Milky Way through a ``Galactic fountain''-like process and therefore comparable metallicity to disk gas.
Both processes clearly occur: the Magellanic stream itself is evidence of gas accretion in to the Milky Way halo, and the well-known Smith Cloud \citep{Smith:1963} is a metal-rich \citep[$\feh \sim 0.5$[]{Fox:2016} HVC that plausibly originated from the Galactic disk \citep[e.g.,][]{Bregman:1980}.
Given the low metallicity of the stars in \clustername, the gas it formed from was likely extragalactic, as any violent star-forming regions in the Milky Way disk that could have driven gas so far out into the halo have, at present-day, significantly higher metallicity than this cluster.

% TODO: Nidever plots at different velocities?
Here we take a closer look at the \hi gas in the vicinity of \clustername.
\figurename~\ref{fig:XXXX} shows the \hi column density in the region around the leading arm of the Magellanic stream, with the location of \clustername\ marked.
While no bulk metallicity measurements exist for the LA gas, \todo{...recent measurements suggest comparable to SMC?} \citep{Fox:2018}.
Given its extragalactic origin, its proximity to the Magellanic stream suggests that \clustername\ could have formed from this gas.
Previously studies have detected young stars in the Leading Arm and gas in the periphery of the Magellanic Clouds \citep{Casetti-Dinescu:2014, MoniBidin:2017}, however, this is the first time that an entire star cluster has been detected so far from the Clouds.

\todo{which gas feature? discuss longitude vs. vlsr}

% We evaluate the hypothesis that the cluster formed in the Leading Arm.  The mean proper motion values in the Magellanic stream coordinate system \citep{Nidever:2008} are ($\mu_{\rm MSL}$,$\mu_{\rm MSB}$)=($+$1.X,0.X) mas yr$^{-1}$ which is very similar to the mean values of the LMC and SMC with ($+$1.X,0.X) mas yr$^{-1}$ and ($+$1.X,0.X) mas yr$^{-1}$, respectively.  The simulations of \citet{Besla:2012} also predict proper motions at this position of approximately ($+$1.X,0.X) mas yr$^{-1}$.  The Magellanic stream simulations give distances of the Leading Arm at this position of $\sim$50 kpc, however, many of these do not include the effects of ram pressure from the hot MW halo which could cause the gas to drop in its orbit to smaller Galactocentric distances (NEED TO DOUBLE-CHECK THIS).  In addition Figure \ref{fig_gass} shows that there is Leading Arm \hi~gas at the position of the position of the cluster with a velocity of \vlsr=70 \kmse.

\todo{Radial velocity follow-up is needed for internal motion, and check association with gas}


\begin{figure}
\centering
\includegraphics[width=8cm]{gass_vlsrmlon.pdf}
\caption{GASS position-velocity diagram.}
\label{fig_gass}
\end{figure}

\begin{figure}
\centering
\includegraphics[width=8cm]{gass_mlatmlon.pdf}
\caption{GASS \hi column density in the region of LA II. The red cross marks the position
of the young cluster.}
\label{fig_gass}
\end{figure}


\subsection{The Galactic orbit of \clustername}
\label{sec:orbit}

\todo{Assume velocity, get orbit. Lower energy than LMC, but expected if dissipated during star formation? How much orbital energy lost? Give orbital parameters at midplane.}

If we assume that \clustername\ is associated with nearby \hi in the Magellanic stream, we have measurements of all 6 phase-space components of the cluster and can compute orbits for \clustername.
However, as discussed in \sectionname~\ref{sec:higas}, there are two prominent but distinct velocity components of \hi along the line of sight to \clustername.
We therefore integrate orbits of the cluster using both values of the line-of-sight velocity from these two gas components.
\figurename~\ref{fig:orbits}



\begin{figure}
\centering
\includegraphics[width=0.5\textwidth]{figures/orbits.pdf}
\caption{TODO:}
\label{fig:orbits}
\end{figure}

% \section{Discussion} \label{sec:discussion}
%
% - Streams or clusters in the halo from star-formation from Sgr or dwarf gas
%
% - Sgr young globular clusters: same thing as this? Indication of when Sgr lost its gas?
%
% - Stars have the right metallicity to be SMC gas, I think
%
% - TODO: where would interaction point between LMC gas and MW disk gas be today? ~150–200 Myr ago is almost 1 complete disk rotation. see Bekki:2008


\section{Conclusion} \label{sec:conclusion}

We have identified a young, metal-poor stellar association in the Galactic halo --- named \clustername --- with an age $\tau \approx \clage$, heliocentric distance $D \approx \cldist$, and metallicity $\feh \approx \clfeh$.
At its present day sky position, and at large distances, all significant quantities of \hi are associated with the leading arm of the Magellanic gas stream and thus it plausibly formed from Magellanic stream gas.
The age of the cluster is broadly consistent with the time it would have most recently crossed the Galactic midplane, suggesting the possibility that interaction with the Milky Way disk or tidal compression could have triggered this star formation event.
The discovery of \clustername\ provides a critical distance constraint to the Magellanic stream and will aid future Magellanic system and Milky Way modeling efforts.
It also provides an opportunity to study star formation in a unique environment, unlike that of the Milky Way disk or any other cluster-forming region.
The serendipitous discovery of this cluster is a reminder that the combined value of the \gaia\ data with deep, large-area imaging surveys provides a wealth of information about our Galaxy and stellar halo.


\acknowledgments

It is a pleasure to thank
Lauren Anderson (Flatiron),
Ana Bonaca (Harvard),
Elena D'Onghia (UW Madison),
Dan Foreman-Mackey (Flatiron),
Raja Guhathakurta (UCSC),
Cliff Johnson (Northwestern),
Ekta Patel (Arizona),
Josh Peek (STScI),
Anil Seth (Utah),
and Erik Tollerud (STScI)
for useful suggestions and discussion.

This work has made use of data from the European Space Agency (ESA)
mission {\it Gaia} (\url{https://www.cosmos.esa.int/gaia}), processed by
the {\it Gaia} Data Processing and Analysis Consortium (DPAC,
\url{https://www.cosmos.esa.int/web/gaia/dpac/consortium}). Funding
for the DPAC has been provided by national institutions, in particular
the institutions participating in the {\it Gaia} Multilateral Agreement.

We thank the Scientific Computing Core at the Flatiron Institute, and especially Dylan Simon and Nick Carriero, for technical support and access to the Flatiron Institute cluster computing resources, which enabled this work.
We thank the Center for Computational Astrophysics and especially David Spergel for support, access to computational resources, and space to conduct this work.


\appendix

\section{Queries}
\label{sec:queries}

Initial query to select very blue stars away from the Galactic plane:
\begin{verbatim}
SELECT * FROM gaiadr2.gaia_source
WHERE parallax < 1
AND (bp_rp > -0.5) AND (bp_rp < 0)
AND phot_g_mean_mag < 20
AND ABS(b) > 20
\end{verbatim}

Query to retrieve \gaia\ data around the blue, comoving group found and discussed in \sectionname~\ref{sec:data}:
\begin{verbatim}
SELECT *
FROM gaiadr2.gaia_source
WHERE (parallax < 1 OR parallax IS NULL)
    AND ra > 177 AND ra < 182
    AND dec > -31.3 AND dec < -26.3
\end{verbatim}

\software{
    \package{Astropy} \citep{astropy, astropy:2018},
    \package{dustmaps} \citep{dustmaps},
    \package{emcee} \citep{emcee, emcee:ascl},
    \package{gala} \citep{gala},
    \package{IPython} \citep{ipython},
    \package{matplotlib} \citep{mpl},
    \package{numpy} \citep{numpy},
    \package{PyMultinest} \citep{Buchner:2014},
    \package{schwimmbad} \citep{schwimmbad},
    \package{scipy} \citep{scipy}
}

\bibliographystyle{aasjournal}
\bibliography{ms}

\end{document}
